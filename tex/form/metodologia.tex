%--------------------------------------------------------------------------------------
% Metodologia a ser empregada 
%--------------------------------------------------------------------------------------
Levantamento bibliográfico, feito com auxílio do Google Scholar \url{https://scholar.google.com.br/} para a pesquisa inicial, sobre Recuperação de Informação, Mineração de Textos, Mineração de Dados, Análise de Textos e Classificação de Autores.
Será utilizado como critério de seleção dos artigos e livros:
\begin{itemize}
    \item Estar escrito em português ou inglês; e
    \item Ter no resumo/sumário quaisquer das palavras-chave: Recuperação de Informação; Mineração de Textos; Mineração de Dados, Análise de Textos; Classificação de Autores; Information Retrieval; Text Mining; Data Mining; Text Analysis; e Autor Profiling; e
    \item Disponibilidade gratuita pela internet, podendo ser encontrado utilizando o Google ou Google Scholar ou disponibilizado pela Library Genesis \url{http://gen.lib.rus.ec/} ou Sci-Hub \url{http://sci-hub.tw/}.
\end{itemize}
Haverá priorização pelos artigos e livros mais recentes, e também pelos que forem mais citados em outros artigos que atendam ao tema.

Os artigos e livros serão categorizados de acordo com as cinco palavras chave estabelecidas acima nos critérios do levantamento, sendo então feita uma revisão livre dos artigos escolhidos como relevantes pelo autor, sendo estes lidos em sua totalidade, anotados e resumidos e devidamente fichados. E os livros encontrados serão inicialmente analisados pelo seu sumário, sendo localizados os capítulos pertinentes aos temas do levantamento, e estes então lidos e também resumidos.

O processo de levantamento bibliográfico será feito interativamente com a escrita do trabalho, e este último também será desenvolvido em paralelo com o processo de estudo posterior das técnicas e ferramentas de RI (Recuperação de Informação) a serem utilizadas como parâmetro em classificadores de MT (Mineração de Texto).

A seleção dos banco de dados a serem aplicados os classificadores de MT será feita após ter sido concluída a revisão bibliográfica e terem sido já avaliados quais grupos de parâmetros, ou variáveis, de RI serão utilizados e quais os respectivos classificadores de MT que vão utilizar dessas variáveis.