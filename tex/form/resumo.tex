%--------------------------------------------------------------------------------------
% Resumo e palavras-chave
%--------------------------------------------------------------------------------------
O mundo informatizado gera uma quantidade gigantesca de dados textuais diariamente, e a Mineração de Texto objetiva transformar esses dados em informações úteis, em conhecimento, tendo aplicação inclusive na área forense.
A área de Recuperação de Informação contribui para o desenvolvimento da Mineração de Texto no pré-processamento, porém sua utilização direta na criação de atributos não é usual, sendo proposto pela primeira vez por \citeonline{WEREN_CLEF_2014}.

A partir de uma revisão bibliográfica das áreas de Recuperação de Informação e de Mineração de Texto, o autor sugere uma metodologia para criação de atributos utilizando a função de ranqueamento BM25, similar à utilizada por \citeonline{WEREN_MESTRADO_2014}, e utilizando ferramentas de armazenamento e indexação já existentes para o cálculo.
Nesta metodologia, a análise do desempenho dos novos atributos sugeridos é feita por meio da mensuração do desempenho de classificador por medidas consolidadas na literatura de Mineração de Texto.


Palavras-chave: Mineração de Texto, Recuperação de Informação, criação de atributos, avaliação de desempenho, engenharia de atributos.